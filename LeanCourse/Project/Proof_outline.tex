\documentclass[11pt]{article}
\usepackage{amsmath,amsthm,amsfonts,amssymb,amscd}
\usepackage{multirow,booktabs}
\usepackage[table]{xcolor}
\usepackage{fullpage}
\usepackage{lastpage}
\usepackage{enumitem}
\usepackage{fancyhdr}
\usepackage{mathrsfs}
\usepackage{wrapfig}
\usepackage{setspace}
\usepackage{calc}
\usepackage{multicol}
\usepackage{cancel}
\usepackage[retainorgcmds]{IEEEtrantools}
\usepackage[margin=3cm]{geometry}
\usepackage{amsmath}
\newlength{\tabcont}
\setlength{\parindent}{0.0in}
\setlength{\parskip}{0.05in}
\usepackage{empheq}
\usepackage{framed}
\usepackage[most]{tcolorbox}
\usepackage{xcolor}
\usepackage{tikz-cd}
\usepackage{hyperref}
\hypersetup{colorlinks=true, linkcolor=teal}
\usepackage[colorinlistoftodos,prependcaption,textsize=tiny]{todonotes}
\swapnumbers
\colorlet{shadecolor}{orange!15}
\parindent 0in
\parskip 12pt
\geometry{margin=1in, headsep=0.25in}
\theoremstyle{definition}
\newtheorem{defin}{Definition}
\newtheorem{proposition}[defin]{Proposition}
\newtheorem{lem}[defin]{Lemma}
\newtheorem{thm}[defin]{Theorem}
\newtheorem{rem}{Remark}[defin]

\newcommand{\sup}{\text{sup}}

\title{A (hopefully formalization-friendly) Proof of Fodor's Lemma}
\author{Theofanis Chatzidiamantis Christoforidis}
\date{}


\begin{document}

\maketitle

\begin{defin}
    Let $\lambda$ be a limit ordinal and $C\subseteq \lambda$ a set.
    \begin{enumerate} \itemsep0.2em
        \item[i.]  $C$ is called \textbf{unbounded} in $\lambda$ if $\sup (C\cap \lambda)=\lambda$.
        \item[ii.] $C$ is called \textbf{closed} in $\lambda$ if for every $\alpha < \lambda$,
        if $C\cap\alpha\neq\varnothing$, then $\sup (C\cap\alpha)\in C$.
        \item[iii.] $C$ is called \textbf{club} in $\lambda$ if it is closed and unbounded
        in $\lambda$.
    \end{enumerate}
\end{defin}

\begin{rem}
    Equivalently, $C\subseteq\lambda$ is unbounded in $\lambda$ if for every $\alpha < \lambda$,
        there is a $\beta\in C$ such that $\alpha<\beta$.
\end{rem}

\begin{defin}
    Let $\lambda$ be a limit ordinal and $S\subseteq \lambda$ a set. $S$ is called
    \textbf{stationary} in $\lambda$ if for every club set $C\subseteq\lambda$,
    $S\cap C\neq\varnothing$.
\end{defin}

Our goal is to give a self-contained (assuming the contents of Mathlib) proof of the following:
\begin{thm}[Fodor's Lemma] \textcolor{orange}{[\textit{regressive\_on\_stationary}]}
   \newline Let $\kappa$ be an uncountable regular cardinal, $S\subseteq\kappa$ a
   stationary set and let $f:S\rightarrow \kappa$ be an ordinal function such
   that $f(\alpha)<\alpha$ for all $\alpha\in S, \alpha>0$. Then there is an ordinal
   $\theta <\kappa$ such that the preimage $f^{-1}(\{\theta\})$ is stationary in $\kappa$.
\end{thm}

\begin{lem}
    The intersection of two club sets is club.
\end{lem}

\begin{proof}
    Let $C,D\subseteq\lambda$ be club in $\lambda$.
    \begin{itemize}
        \item To show that $C\cap D$ is closed in $\lambda$, assume $\alpha < \lambda$ and
        that $C\cap D\cap \alpha \neq\varnothing$. Let $\beta=\text{sup}(C\cap D\cap\alpha)$.
        Then $\text{sup}(C\cap \beta)=\text{sup}(C \cap \text{sup}(C\cap D\cap\alpha))
        = \text{sup}(C\cap D\cap\alpha) = \beta$, and thus $\beta\in C$ as $C$ is club.
        Similarly $\text{sup}(D\cap\beta)=\beta$ and $\beta\in D$, meaning $\beta\in C\cap D$.
        \item We now show that $C\cap D$ is unbounded in $\lambda$.
    \end{itemize}
\end{proof}

\begin{proposition}\label{inlt} \textcolor{orange}{[\textit{int\_lt\_card\_club}]}
    Let $\kappa$ be an uncountable regular cardinal and let $(C_\alpha)_{\alpha<\lambda}$ be a sequence of
    subsets of $\kappa$ where $\lambda < \kappa$. If every $C_\alpha$ is club in $\kappa$, then the
    intersection $\cap_{\alpha < \lambda}C_{\alpha}$ is club in $\kappa$.
\end{proposition}

\begin{proof}

\end{proof}

\begin{defin}\textcolor{orange}{[\textit{diag\_int}]}
    Let $\kappa$ be a cardinal and let $(C_\alpha)_{\alpha <\kappa}$ be a sequence of subsets of
    $\kappa$. The \textbf{diagonal intersection} of $(C_\alpha)_{\alpha <\kappa}$ is defined to be
    $$\Delta_{\alpha<\kappa}C_\alpha:=\{\beta <\kappa\ |\ \beta \in C_{\theta}\ \forall\theta<\beta\}$$
\end{defin}

\begin{lem}\label{diagint} \textcolor{orange}{[\textit{diag\_int\_club}]}
    Let $\kappa$ be a cardinal and let $(C_\alpha)_{\alpha <\kappa}$ be a sequence of subsets of
    $\kappa$. If every $C_{\alpha}$ is club in $\kappa$, then $\Delta_{\alpha<\kappa}C_\alpha$ is club in
    $\kappa$.
\end{lem}

\begin{proof}

\end{proof}

\begin{proof}[Proof of Fodor's Lemma.]
    Assume, towards a contradiction, that there exists no such $\theta$, i.e., $$\{\alpha\in S: f(\alpha)
    =\beta\}$$ is not stationary for any $\beta<\kappa$. Then, for every $\beta<\kappa$, we can choose a
    club set $C_\beta$ satisfying $f(\alpha)\neq\beta$ for every $\alpha\in S\cap C_\beta$. By (\ref{diagint}),
    $C:=\Delta_{\beta<\kappa}C_{\beta}$ is club in $\kappa$. Pick an $\alpha\in C\cap S$, which is nonempty
    since $C$ is club and $S$ is stationary in $\kappa$. By definition of the diagonal intersection, this
    means that $f(\alpha)\neq\beta$ for every $\beta < \alpha$. This implies $f(\alpha)\geq\alpha$,
    contradicting the assumption that $f$ is regressive on $S$.
\end{proof}

\begin{thebibliography}{}
    \bibitem{Jech03} Jech, Thomas. Set Theory: The Third Millennium Edition, Springer, 2003.
    \bibitem{Sch} Schimmerling, Ernest. A Course on Set Theory, Cambridge University Press, 2011.
\end{thebibliography}

\end{document}