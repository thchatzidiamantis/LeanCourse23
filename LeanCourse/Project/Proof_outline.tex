\documentclass[11pt]{article}
\usepackage{amsmath,amsthm,amsfonts,amssymb,amscd}
\usepackage{multirow,booktabs}
\usepackage[table]{xcolor}
\usepackage{fullpage}
\usepackage{lastpage}
\usepackage{enumitem}
\usepackage{fancyhdr}
\usepackage{mathrsfs}
\usepackage{wrapfig}
\usepackage{setspace}
\usepackage{calc}
\usepackage{multicol}
\usepackage{cancel}
\usepackage[retainorgcmds]{IEEEtrantools}
\usepackage[margin=3cm]{geometry}
\usepackage{amsmath}
\newlength{\tabcont}
\setlength{\parindent}{0.0in}
\setlength{\parskip}{0.05in}
\usepackage{empheq}
\usepackage{framed}
\usepackage[most]{tcolorbox}
\usepackage{xcolor}
\usepackage{tikz-cd}
\swapnumbers
\colorlet{shadecolor}{orange!15}
\parindent 0in
\parskip 12pt
\geometry{margin=1in, headsep=0.25in}
\theoremstyle{definition}
\newtheorem{defin}{Definition}
\newtheorem{prop}[defin]{Proposition}
\newtheorem{lem}[defin]{Lemma}
\newtheorem{thm}[defin]{Theorem}
\newtheorem{rem}{remark}[defin]

\title{A (hopefully formalization-friendly) Proof of Fodor's Lemma}
\author{Theofanis Chatzidiamantis Christoforidis}
\date{}


\begin{document}

\maketitle

\begin{defin}
    Let $\lambda$ be a limit ordinal and $C\subseteq \lambda$ a set.
    \begin{enumerate} \itemsep0.2em
        \item[i.]  $C$ is called \textbf{unbounded} in $\lambda$ if $sup(C\cap \lambda)=\lambda$.
        \item[ii.] $C$ is called \textbf{closed} in $\lambda$ if for every $\alpha < \lambda$,
        if $C\cap\alpha\neq\varnothing$, then $sup(C\cap\alpha)=\alpha$.
        \item[iii.] $C$ is called \textbf{club} in $\lambda$ if it is closed and unbounded
        in $\lambda$.
    \end{enumerate}
\end{defin}

\begin{defin}
    Let $\lambda$ be a limit ordinal and $S\subseteq \lambda$ a set. $S$ is called
    \textbf{stationary} in $\lambda$ if for every club set $C\subseteq\lambda$,
    $S\cap C\neq\varnothing$.
\end{defin}

Our goal is to give a self-contained (assuming the contents of Mathlib) proof of the following:

\begin{thm}[Fodor's Lemma]
   Let $\kappa$ be an uncountable regular cardinal, $S\subseteq\kappa$ be a
   stationary set and let $f:S\rightarrow \kappa$ be an ordinal function such
   that $f(\alpha)<\alpha$ for all $\alpha\in S, \alpha>0$. Then there is a
   stationary set $T\subset S$ and some $\theta <\kappa$ such that
   $f(\alpha)=\theta \ \text{for all} \ \alpha\in T.$
\end{thm}

\begin{thebibliography}{}
    \bibitem{} Jech, Thomas. Set Theory: The Third Millennium Edition, Springer, 2003.
    \bibitem{} Schimmerling, Ernest. A Course on Set Theory, Cambridge University Press, 2011.
\end{thebibliography}

\end{document}