\documentclass{beamer}
\usetheme{Dresden}
\usepackage{ucs}
\usepackage[utf8x]{inputenc}
\usepackage{kerkis}
\usepackage{multicol}
\usepackage{paralist}
\usepackage{graphicx}
\usepackage{amssymb}
\usepackage{amsthm}
\swapnumbers
\usepackage{amsmath}
\usepackage{mathtools}
\usepackage{tikz-cd}
\usetikzlibrary{decorations.pathmorphing, decorations.text}
\usepackage{graphicx}

\theoremstyle{definition}
\newtheorem{defin}[subsection]{Definition}
\newtheorem{prop}[subsection]{Proposition}
\newtheorem{cor}[subsection]{Corollary}
\newtheorem{lem}[subsection]{Lemma}
\newtheorem{thm}[subsection]{Theorem}
\newtheorem{rem}[subsection]{Remark}
\newtheorem{ex}[subsection]{Example}

\title{Formalization of the Proof of Fodor's Lemma}
\author{Theofanis Chatzidiamantis Christoforidis}
\date{January 2024}

\begin{document}

\maketitle

\section{The Math}

\begin{frame}{Basic Definitions}
    Let $o$ be a limit ordinal and $C \subseteq o$ a set.
    \begin{defin}
        \begin{itemize} \itemsep0.2em
        \item $C$ is called \textbf{unbounded} in $o$ if $\sup (C\cap o)=o$.
        \pause
        \item $C$ is called \textbf{closed} in $o$ if for every $\alpha < o$, if $C\cap\alpha\neq\varnothing$, then $\sup (C\cap o)\in C$.
        \pause
        \item $C$ is called \textbf{club} in $o$ if it is closed and unbounded in $o$.
        \end{itemize}
    \end{defin}
    \pause
    \begin{rem}
        Equivalently, $C$ is unbounded in $o$ if for every $\alpha < o$, there is a $\beta\in C$ such that $\alpha<\beta$.
    \end{rem}
\end{frame}

\begin{frame}{Basic Definitions}
    \begin{defin}
    Let $o$ be a limit ordinal and $S\subseteq o$ a set. $S$ is called \textbf{stationary} in $o$ if for every club set $C\subseteq o$, $S\cap C\neq\varnothing$.
\end{defin}
\pause
\begin{defin}
    Let $f:C\to D$ be a function, where $C$ and $D$ are sets of ordinals. $f$ is callted \textbf{regressive} if for every $0 < \alpha \in C$, $f(\alpha) < \alpha$.
\end{defin}
\end{frame}

\begin{frame}{Basic Definitions}
    \begin{defin}
        Let $\kappa$ be a cardinal.
        \begin{itemize}
            \item The \textbf{cofinality} $cof(\kappa)$ of $\kappa$ is the smallest possible cardinality of an unbounded set in $\kappa$ \pause
            \item A cardinal is \textbf{regular} if $cof(\kappa)=\kappa$
        \end{itemize} \pause
        \begin{rem}
            Cofinality generalizes to ordinals and this is what Lean actually does.
        \end{rem}
    \end{defin}

\end{frame}

\begin{frame}{The Theorem}
    Our goal is to prove the following:
    \begin{thm}[Fodor's Lemma]
       Let $\kappa$ be an uncountable regular cardinal, $S\subseteq\kappa$ a stationary set and let $f:S\rightarrow \kappa$ be a regressive function. Then there is an ordinal $\theta <\kappa$ such that $f^{-1}(\{\theta\})$ is stationary in $\kappa$.
    \end{thm}
\end{frame}

\section{Translating to Lean}
%Say how you did strict res
%Say why we are taking function to the type Ordinal and restricting afterwards, set theory vs type theory
%Constructivists avert your eyes

\begin{frame}{Sets and Ordinals are different types.}
    \begin{itemize}
        \item In set theory, everyting is a set, including cardinals and ordinals. This is not true in the type-theoretic framework we are working with. \pause
        \item With this in mind, our main objects of study here are of type \textit{\textbf{Set Ordinal}}. \pause
        \item This makes sense: recall that in ZFC, an ordinal contains every ordinal strictly smaller than itself.
    \end{itemize}
\end{frame}

\begin{frame}{Sets and Ordinals are different types.}
    In Lean, for C : \textit{Set Ordinal} and o : \textit{Ordinal}, we define:
    \begin{itemize}
        \item \text{strict\_Ordinal\_res C o} := $\{c\in C | c < o\}$, corresponding to $C\cap o.$ \pause
        \item \text{Ordinal\_res C o} := $\{c\in C | c \leq o\}$ \pause
    \end{itemize}
    "strict" is the correct notion, but Ordinal\_res has nicer properties and we use it whenever they are equal (i.e., when $o\notin C$)
\end{frame}

\section{Comments on the Proof}

\begin{frame}
    We need one last definition : \pause
    \begin{defin}
        Let $\kappa$ be a cardinal and let $(C_\alpha)_{\alpha <\kappa}$ be a sequence of subsets of
    $\kappa$. The \textbf{diagonal intersection} of $(C_\alpha)_{\alpha <\kappa}$ is defined to be
    $$\Delta_{\alpha<\kappa}C_\alpha:=\{\beta <\kappa\ |\ \beta \in C_{\theta}\ \forall\theta<\beta\}$$
    \end{defin}
\end{frame}

\begin{frame}{General Structure:}
    \[\begin{tikzcd}
        {\text{Basic properties of supremums + Definitions + regularity of} \ \kappa} \\
        {\text{The intersection of two club sets is club}} \\
        {\text{The intersection of less than} \ \kappa \ \text{club sets is club}} \\
        {\text{The diagonal intersection of} \ \kappa \ \text{club sets is club}} \\
        {\text{Fodor's Lemma}} \\
        {}
        \arrow["{\text{induction}}", squiggly, from=2-1, to=3-1]
        \arrow[squiggly, from=3-1, to=4-1]
        \arrow[squiggly, from=4-1, to=5-1]
        \arrow[squiggly, from=1-1, to=2-1]
    \end{tikzcd}\]
\end{frame}

\begin{frame}{Proving Closedness:}
    \pause This came down to manipulating supremums.
\end{frame}

\begin{frame}{Proving Unboundedness:}
    \begin{itemize}
        \item[] For each of the main theorems about club sets, we construct a "good" increasing sequence $\mathbb{N}\to Ordinal$ recursively. \pause
        \item[] We then use our closedness assumptions to prove that its limit is in our intersection.
        \item[] These arguments use both closedness and unboundedness assumptions.
    \end{itemize}
\end{frame}

\begin{frame}{On Cofinality}
    \begin{itemize}
        \item[] Why is it useful to work with sequences of length $\omega$? \pause
        \item[] If $\kappa$ is uncountable and regular, the limit of such an ascending sequence will always be less than $\kappa$ by regularity of $\kappa$, since $\mathbb{N}$ is countable.
    \end{itemize} \pause
    \par With this in mind, a very easy generalisation for the properties of club sets is apparent:
    \begin{rem}
        We can replace our uncountable regular cardinal with any ordinal $o$ satisfying $cof(o)>\aleph_0$.
    \end{rem}
\end{frame}

\end{document}
