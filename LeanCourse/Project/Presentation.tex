\documentclass{beamer}
\usetheme{Szeged}
\usepackage{ucs}
\usepackage[utf8x]{inputenc}
\usepackage{kerkis}
\usepackage{multicol}
\usepackage{paralist}
\usepackage{graphicx}
\usepackage{amssymb}
\usepackage{amsthm}
\swapnumbers
\usepackage{amsmath}
\usepackage{mathtools}
\usepackage{tikz-cd}
\usepackage{graphicx}

\theoremstyle{definition}
\newtheorem{defin}[subsection]{Definition}
\newtheorem{prop}[subsection]{Proposition}
\newtheorem{cor}[subsection]{Corollary}
\newtheorem{lem}[subsection]{Lemma}
\newtheorem{thm}[subsection]{Theorem}
\newtheorem{rem}[subsection]{Remark}
\newtheorem{ex}[subsection]{Example}

\title{Formalization of the Proof of Fodor's Lemma}
\author{Theofanis Chatzidiamantis Christoforidis}
\date{January 2024}

\begin{document}

\maketitle

\section{The Math}

\begin{frame}{Basic Definitions}
    Let $o$ be a limit ordinal and $C \subseteq o$ a set.
    \begin{defin}
        \begin{enumerate} \itemsep0.2em
        \item[i.]  $C$ is called \textbf{unbounded} in $o$ if $\sup (C\cap o)=o$.
        \pause
        \item[ii.] $C$ is called \textbf{closed} in $o$ if for every $\alpha < o$, if $C\cap\alpha\neq\varnothing$, then $\sup (C\cap o)\in C$.
        \pause
        \item[iii.] $C$ is called \textbf{club} in $o$ if it is closed and unbounded in $o$.
        \end{enumerate}
    \end{defin}
    \pause
    \begin{rem}
        Equivalently, $C$ is unbounded in $o$ if for every $\alpha < o$, there is a $\beta\in C$ such that $\alpha<\beta$.
    \end{rem}
\end{frame}

\begin{frame}{Basic Definitions}
    \begin{defin}
    Let $o$ be a limit ordinal and $S\subseteq o$ a set. $S$ is called \textbf{stationary} in $o$ if for every club set $C\subseteq o$, $S\cap C\neq\varnothing$.
\end{defin}
\pause
\begin{defin}
    Let $f:C\to D$ be a function, where $C$ and $D$ are sets of ordinals. $f$ is callted \textbf{regressive} if for every $0 < \alpha \in C$, $f(\alpha) < \alpha$.
\end{defin}
\end{frame}

\begin{frame}{The Theorem}
    Our goal is to prove the following:
    \begin{thm}[Fodor's Lemma]
       Let $\kappa$ be an uncountable regular cardinal, $S\subseteq\kappa$ a stationary set and let $f:S\rightarrow \kappa$ be a regressive function. Then there is an ordinal $\theta <\kappa$ such that $f^{-1}(\{\theta\})$ is stationary in $\kappa$.
    \end{thm}
\end{frame}

\section{Translating to Lean}
%Say how you did strict res
%Say why we are taking function to the type Ordinal and restricting afterwards, set theory vs type theory
%Constructivists avert your eyes

\begin{frame}{Sets and Ordinals are different types. How can they interact?}
    \begin{itemize}
        \item In set theory, everyting is a set, including cardinals and ordinals. This is not true in the type-theoretic framework we are working with. \pause
        \item With this in mind, our main objects of study here are of type \textbf{Set Ordinal}. \pause
        \item This makes sense: recall that in ZFC, an ordinal contains every ordinal strictly smaller than itself.
    \end{itemize}
\end{frame}

\begin{frame}{Sets and Ordinals are different types. How can they interact?}
    fdsafdsaf
\end{frame}

\begin{frame}{Sets and Ordinals are different types. How can they interact?}
    fdasfdaf
\end{frame}

\begin{frame}{Working with supremums is hard.}
    fdsafda
\end{frame}

\section{Some Proof Examples}

\begin{frame}{gdsjhfhkdjs}

\end{frame}

\end{document}
